  \section{Motivation}
    The present project has been developed within the context of HELICoiD (HypErspectraL Imaging Cancer Detection), which is an European project funded by the Seventh Framework Programme of the European Union and, especially, by the FET - Open (Future \& Emerging Technologies) initiative. Several institutions from different countries such as France, United Kingdom and The Netherlands have been involved in this European project as well. Furthermore, it includes two hospitals, three companies and four universities (Spain included).  This research, particularly, is a collaborative work with UPM (Spain) within research design group of Electronic and Microelectronic.

The main aim of the HELICoiD European project is to provide to the surgeon a technique which informs accurately about healthy tissue and tumours in real time. This is all thanks to Hyperspectral images since traditional methods have low level in terms of sensitivity and the boundaries of the image are not clearly defined. In other words, HELICoiD aims at distinguishing between healthy tissue and tumours by extracting the spectral information of each pixel.It can be assumed that the spectral information is correlated with the chemical composition of a particular material. Therefore, each hyperspectral pixel has a spectral signature of a specific substance.

With regard to this line of research, the present work implements an unsupervised clustering method called K-Means on a parallel architecture in order to supply information in real time to surgeons. 

In this regard, a dataflow language called $\pi SDF$ is used, in order to perform the parallelization of this algorithm. The $\pi SDF$ is a generalization of SDF MoC, is a syncronous dataflow model of computation. An application is modelled by directed graph of computational entities, called actors, that exchange data packets called data tokens, through a
network of First-In First-Out queues (FIFOs)\cite{lee1987synchronous}

The procedure is as follows; hyperspectral (HS) sensors attain hyperspectral cubes, and HS cubes are pre-processed in order to reduce dimensionality and noise. Afterwards, they are clustered employing K-Means, which defines different areas properly. After using this algorithm, an unsupervised segmentation map is generated. 
Meanwhile in parallel, the system executes a number of algorithms belonging to supervised classification. These algorithms are PCA (Principal Components Analysis), SVM (Support Vector Machine) and KNN (K-Nearest Neighbour). After performing these algorithms, tissues are displayed using different colours in order to represent the associated classes. 
Applying the majority voting, the unsupervised segmentation map obtained from K-Means clustering algorithm as well as the classification map obtained from supervised classification, are merged. 

The implementation of this algorithm is carried out using a dataflow specification tool called PREESM. This tool is widely used for manycore architectures and signal processing applications. The objective of parallelizing this algorithm is to speed up computations for data clustering to target real time response.

    \section{Objectives}
    
As was mentioned earlier, the main objective of this project is to analyse a clustering algorithm called K-means in order to find possible parallelization methods and approach real time . The following points have been developed to achieve the global aim:
 \begin{itemize}
\item  Research how to efficiently parallelize an algorithm by considering the bottlenecks that generate delays on the execution of the algorithm. 
\item Study in depth the unsupervised clustering algorithm, especially, an optimized model for hyper-spectral images provided by Universidad de Las Palmas De Gran Canaria.
\item  Learn how to parallelize the chosen algorithm using a dataflow specification tool called PREESM. Ones of the great advantages of this dataflow used in PREESM is the flexibility, predictability and expressivity provided,since this semantic is based on interfaces that fix the number of tokens consumed/produced by a hierarchical vertex.
\item Test the reached speedup by comparing it with the sequential implementation one.
\end{itemize}