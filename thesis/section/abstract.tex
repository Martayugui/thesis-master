In the 21st century, artificial intelligence has shown an upward trend of growth as one of the most important area in computer science. Furthermore, this field has had a high social impact on medicine providing real-time medical diagnoses through image processing and clinical trials. 

The present research work is developed within the context of HELICoiD. Furthermore, HELICoiD project exploits hyperspectral images as they are used along with machine learning for detecting and delineating tumours in brain tissues through a processing toolchain. This toolchain consists of supervised as well as unsupervised methods. 

The accurate delimitation of brain cancer is an important task having a surgery. Several techniques are used in order to guide neurosurgeons in the removal of the tumour.In this approach, the K-Means clustering algorithm aims at the accurate delimitation of the tumour borders. 

The implementation of this algorithm is performed using a dataflow specification tool called PREESM. This tool uses the Parameterized and Interfaced Synchronous Dataflow ($\pi SDF$) MoC model which is a hierarchical and dynamically reconfigurable extension of the SDF MoC. This model is known for their analyzability, their predictability and their natural expressivity of task parallelism in signal processing algorithms. The objective of parallelizing this algorithm is to speed up computations for data clustering to target real time response.In order to achieve this objective, the atomic operations which consume most of the execution time are firstly parallelized.

The results achieved are accurately analysed and validated using an in-vivo hyperspectral human brain image database. Experimental results illustrates that the PREESM version can reach speed-up of    than the secuantial's implementation one. Up to our knowledge, the work done in this thesis approaches real time during surgery as possible.

