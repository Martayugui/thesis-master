En el siglo XXI, la inteligencia artificial muestra una tendencia ascendiente de crecimiento como uno de los campos más importantes en las ciencias computacionales. Además, este campo ha tenido un alto impacto social en medicina, proporcionando diagnósticos al paciente en tiempo real a través del procesado de imágenes y ensayos clínicos.

	El presente trabajo de investigación se desarrolla dentro del contexto de HELICoiD. Además, el proyecto de HELICoiD explota las imágenes hiperespectrales ya que éstas se usan en aprendizaje de máquina para detectar y delinear tumores en el cerebro a través de una cadena de procesado. Esta cadena de procesado está compuesta tanto por aprendizaje supervisado como por aprendizaje no supervisado.
	
	La precisa delimitación de tumores malignos es una tarea crucial durante una cirugía. Para la extracción del tumor, el neurocirujano es guiado a través de multitud de técnicas. En este enfoque, el algoritmo de agrupamiento llamado K-Means tiene como objetivo la correcta delineación de los bordes del tumor.
	
	La implementación de este algoritmo se desarrolla usando un tipo de modelo de flujo de datos llamado PREESM. Esta herramienta usa el modelo síncrono parametrizado e interconectado de flujo de datos que son extensiones reconfigurables y dinámicas del SDF MoC. Este modelo es conocido por su análisis, predictividad y expresividad natural de las tareas paralelizadas en algoritmos de procesado de imágenes. El objetivo de paralelizar este algoritmo es acelerar cómputos para la agrupación de los datos y de esta forma acercarse lo más posible al tiempo real. Con el fin de lograr este objetivo, las operaciones atómicas que consumen la mayor parte de tiempo de ejecución, serán paralelizadas en primer lugar.
	
	Los resultados alcanzados, son analizados de forma precisa y validados usando un conjunto de imágenes hiperespectrales in-vivo del cerebro humano. Los resultados experimentales ilustran que la versión llevada a cabo en PREESM puede alcanzar velocidades veces que la implementación secuencial. Hasta lo que sabemos, el trabajo desarrollado en esta tesis se acerca lo más posible al tiempo real durante las cirugias.
	
	
	
	
	
