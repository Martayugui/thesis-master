    The number of internet connected devices has already by far surpassed the number of human beings.
    The pace of growth is still so big that in the next five years that number will double.
    The ecosystem of these devices, collectively called \ac{IoT}, is a source of a tremendous amount of data and creates several unheard challenges for researchers and companies.
    New, unconventional ways of storing, analyzing, and processing of the data had to be proposed. 
    One such a solution is \ac{Hops}, a result of years-long research between KTH Royal Institute of Technology in Stockholm and RISE SICS AB.
     It is a platform enabling an analysis of extremely large volumes of data with cutting-edge, open-source technologies for Big Data and \ac{ML}.
     This master thesis provides support for connecting these two environments.
     It provides instruments for secure and reliable ingestion of \ac{IoT} data into Hops platform. 
     Moreover, it provides tools for ensuring the level of security by supporting the execution of mitigating measures, such as automated exclusion of misbehaving devices and dropping traffic from sources of \ac{DDoS} attacks. 
     To allow the data ingestion a new element was introduced to the ecosystem - IoT Gateway. 
     It is a platform, where the authenticated \ac{IoT} devices can stream data to. 
     Furthermore, Hopsworks, one of the Hops' main component, was extended with REST API that allowed the gateways to securely connect to the Hops ecosystem. 
     A testbed, including \ac{IoT} software simulator and a real \ac{IoT} device with dedicated hardware, was built and comprehensively tested and benchmarked.
     The architecture is based on the publicly open and very popular security protocols - \ac{RPK} and \ac{HTTPS}.
     It is shown that the proposed solution is performant, scalable, and provides high reliability in a real-life case scenario.
     Up to our knowledge, the work done in this thesis makes Hopsworks the world's first open source Big Data platform with secure \ac{IoT} data ingestion. 